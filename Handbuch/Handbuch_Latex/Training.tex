\section{Antrainieren einer Fernbedienung}\label{sec_trainIr}
		Wird nach einem Neustart der Uhr, w�hrend des Blinkens der Minutenpunkte, 
		ein Infrarot"=Fernbedienungs"=Kommando erkannt, 
		wechselt die Uhr automatisch in den Fernbedienungs"=Trainingsmodus. 
		Nun m�ssen nacheinander allen Kommandos die gew�nschten Fernbedienungstasten 
		zugewiesen werden. 
		
		Soll ein oder mehrere Kommandos nicht zugewiesen werden,
		\ifthenelse{\WCindividualCfg=1}{
		  da diese nicht ben�tigt werden,	
		}{
			weil zum Beispiel keine Funkuhr-Funktion oder kein Ambilight enthalten ist, 
			oder nur einfarbige LEDs (kein RGB) verwendet werden, 
		}
		kann die Option mit erneutem Druck auf die an/aus-Taste �bersprungen werden.		
		Die �bersprungene Option steht dann sp�ter nicht zur Verf�gung, 
		kann jedoch in einer weiteren Anlernprozedur nach einem Neustart wieder aktiviert 
		werden. 
		
		Die Stundenw�rter zeigen an, welches Kommando nun einer Fernbedienungstaste
		zugewiesen werden soll. 
		Nach dem zw�lften Kommando beginnt die Anzeige der Stundenw�rter wieder bei eins.  
		
		\begin{Anmerkung}
				Eine �bersicht �ber die Kommandos und deren Reihenfolge 
				befindet sich am Ende dieses Kapitels.
		\end{Anmerkung}
		
		
		\begin{Anmerkung}
				Aufgrund der Vielzahl an Herstellern von fernbedienbaren Ger�ten 
				existiert eine ebenso gro�e Vielzahl von Fernbedienungsprotokollen. 
				Es wurde versucht, die h�ufigsten Protokolle zu 
				implementieren. Aufgrund der Vielfalt an 
				unterschiedlichen Fernbedienungsprotokollen, kann es 
				vorkommen, dass nicht alle Fernbedienungen erkannt 
				werden. \\
				Sollte w�hrend des Anlernens der Fernbedienung auffallen, 
				dass die Uhr nicht wie erwartet reagiert 
				(die Zahlenanzeige �berspringt Zahlen oder eine Taste wird 
				erst nach mehrmaligen Dr�cken erkannt), 
				kann das Protokoll der verwendeten Fernbedienung nicht 
				problemlos erkannt werden. 
				In einem solchen Fall, ist die Uhr vom Strom zu trennen und 
				mit einer anderen Fernbedienung neu zu beginnen. 
		\end{Anmerkung}


		\subsection{�bersicht der m�glichen Kommandos}
		\begin{minipage}{\textwidth}
		\newcounter{ircmdNumber} 
		\newcommand{\ircCmdTableLine}[1]{  {\stepcounter{ircmdNumber}\theircmdNumber.}& {#1}&{}\tabularnewline\hline}
		\begin{center}
		   % \footnotesize
			  \arrayrulecolor{colorBack}
				\begin{tabular}{|R{.05\textwidth}L{.6\textwidth}|L{.25\textwidth}|}
						\hline
						\rowcolor{colorBack}  {\textcolor{white}{Nr.}} 
																& {\textcolor{white}{Kommando}} 
																& {\textcolor{white}{Taste eigene Fernbedienung}} 
																\tabularnewline\hline
						\ircCmdTableLine{ein/aus}
						\ircCmdTableLine{heller}
						\ircCmdTableLine{dunkler}
						\ircCmdTableLine{rauf/+}
						\ircCmdTableLine{runter/-}
						\ircCmdTableLine{manuelle Zeiteingabe}
						\ircCmdTableLine{Eingabe der Ein-/Ausschaltzeit}
						\ifthenelse{\WCindividualCfg=0 \or \WCdcf=1}{
							\ircCmdTableLine{Funkuhr-Abgleich manuell starten}
						}{}
						
      			\ircCmdTableLine{\normalModeName \ifthenelse{\WCindividualCfg=0 \or \WCmonocolor=0}{/Farbprofile aktivieren}{}}
      			\ircCmdTableLine{Pulsierender Modus}
      			\ircCmdTableLine{Demo-Modus}
		        \ifthenelse{\WCindividualCfg=0 \or \WCmonocolor=0}{
		      			\ircCmdTableLine{Automatischer Farbwechselmodus (Regenbogen)}
		      			\ircCmdTableLine{Farbe Rot �ndern}
		      			\ircCmdTableLine{Farbe Gr�n �ndern}
		      			\ircCmdTableLine{Farbe Blau �ndern}
		      			\ircCmdTableLine{Farben durchschalten (Regenbogen/Hue)}
						}{}
      			\ircCmdTableLine{aktuelle Helligkeit �bernehmen}
						\ifthenelse{\WCindividualCfg=0 \or \WCambilight=1}{ 
		      			\ircCmdTableLine{Ambilight ein-/ausschalten}
		        }{}
						\ifthenelse{\WCindividualCfg=0 \or \WCbluetooth=1}{ 
		      			\ircCmdTableLine{Bluetooth ein-/ausschalten}
		        }{}
						\ifthenelse{\WCindividualCfg=0 \or \WCauxIO=1}{ 
		      			\ircCmdTableLine{zus�tzlichen IO ein-/ausschalten}
		        }{}
      			\ircCmdTableLine{Umschalten zwischen den Regional-Modi}
%      			\ircCmdTableLine{}
				\end{tabular}
		\end{center}		
		
		\end{minipage}