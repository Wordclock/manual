
\section{Uhrzeit}\label{sec_time}
		Die WordClock verf�gt �ber eine integrierte Quarzuhr,
		welche eine �u�erst genaue Zeitanzeige erm�glicht. 
		
		Optional enth�lt die WordClock einen Funkuhr"=Empf�nger, 
		der die Uhr regelm��ig mit der genauen Uhrzeit versorgt. 
		Damit wird zum Beispiel auch eine manuelle Umstellung von 
		Sommer- und Winterzeit �berfl�ssig. 
		
		Die interne Quarzuhr verf�gt �ber eine kleine 
		Pufferbatterie, die im Falle einer Trennung vom 
		Stromnetz daf�r sorgt, dass die Uhrzeit nicht verloren 
		geht und normal weiterl�uft. 
		
		Nach einem Neustart der Uhr steht so die aktuelle Uhrzeit 
		auch dann zur Verf�gung, wenn kein Funk-Uhr-Empf�nger enthalten ist.
		
		\subsection{Uhrzeit manuell einstellen}\label{sec_time_manualSet}
				Mit der Taste "`Uhrzeiteingabe"' kann die Zeit manuell eingegeben werden.
				  
				Nach dem ersten Druck auf die zugeordnete Taste, blinkt die Stundenanzeige.
				Mittels der Hoch/Runter-Tasten kann nun die Stunde eingestellt werden. 
				
				F�r die Stunden 20\,Uhr\,--\,8\,Uhr leuchten die LEDs mit verminderter 
				und f�r die Stunden 8\,Uhr\,--\,20\,Uhr mit voller Helligkeit, 
				um Tag oder Nacht zu signalisieren. 
				
				So ist es trotz fehlender 24h-Anzeige m�glich 
				zwischen Tag- und Nachtstunden zu unterscheiden. 
				
				Nach einem weiteren Druck auf die zugeordnete Taste, 
				blinken die Minuten, die ebenfalls mit den 
				"`Hoch/Runter"' Tasten eingestellt werden k�nnen.
				  
				Ein dritter Druck �bernimmt dann die eingestellte Zeit 
				und wechselt in den zuvor verlassenen Modus. 

    \ifthenelse{\WCindividualCfg=0 \or \WCdcf=1}
    {
			\subsection{Funkuhr-Abgleich manuell starten}\label{sec_time_manualDCF}
					Mit der Taste "`Funkuhr"=Abgleich"' wird ein manueller 
					Abgleich der Zeit �ber das Funkuhr-Modul initiiert.  
					
					Der Abgleich kann, je nach Empfangsqualit�t, mehrere 
					Minuten dauern. 
					
					In normalen Betrieb erfolgt der Abgleich jeweils zur 
					vollen Stunde. 
    }
		
		\subsection{Automatische An-/Ausschaltzeit}\label{sec_time_setOnOff}
				Hiermit k�nnen Zeiten festgelegt werden, in denen die LEDs der Uhr 
				abgeschaltet werden um einerseits Strom zu sparen und andererseits die 
				Lebensdauer der LEDs zu erh�hen. 
				
				
				Es m�ssen zwei Zeiten eingegeben werden: die 
				Ausschaltzeit und die Einschaltzeit. Sind diese Zeiten 
				identisch, ist die automatische Abschaltung 
				deaktiviert. 
				
				Die Eingabe der Uhrzeiten funktioniert analog zu der 
				Eingabe der Systemzeit. Mit dem Unterschied, dass 
				zwei Zeiten eingegeben werden m�ssen und zur 
				Best�tigung die Taste "`An/Ausschalt"=Zeiteingabe"' 
				bet�tigt werden muss. 
				
				Anschlie�end kann noch ausgew�hlt werden, ob eine Animation 
				(abwechselndes Leuchten der Minuten-LEDs) w�hrend des Auto-Aus-Modus
				angezeigt werden soll.
				
				Damit ergibt sich folgende Prozedur: 
				\begin{compactitem}
						\item  Dr�cken der "`An/Ausschalt-Zeiteingabe"' Taste 
						\item  Stundeneingabe Abschaltzeit 
						\item  Dr�cken der "`An/Ausschalt-Zeiteingabe"' Taste 
						\item  Minuteneingabe Abschaltzeit (10min-Schritten) 
						\item  Dr�cken der "`An/Ausschalt-Zeiteingabe"' Taste 
						\item  Stundeneingabe Einschaltzeit 
						\item  Dr�cken der "`An/Ausschalt-Zeiteingabe"' Taste 
						\item  Minuteneingabe Einschaltzeit (10min-Schritten) 
						\item  Dr�cken der "`An/Ausschalt-Zeiteingabe"' Taste 
						\item  Mit "`Hoch/Runter"' Auto-Aus-Signalisierung w�hlen:
						       1 = Aus, 2 = An
						\item  Dr�cken der "`An/Ausschalt-Zeiteingabe"' Taste 
				\end{compactitem}
				
				Anschlie�end wird wieder in den urspr�nglichen Modus 
				gewechselt. 
				
				Hinweise zur Verhaltensweise
				\begin{compactitem}
				%\item Im Auto-Aus-Modus leuchtet nur abwechelnd eine der Minuten-LEDs zur Signalisierung.
				%\item Jeweils zum Minutenwechsel erscheint die neue Zeit kurzzeitig.
				\item Wurde die Uhr manuell abgeschalten, 
				      wird sie beim Erreichen der Einschaltzeit nicht aktiviert.			
				\item Wird die Uhr im Auto-Aus-Modus manuell eingeschalten,
				      wird sie beim Erreichen der n�chsten Auschaltzeit trotzdem wieder abgeschalten.
				\end{compactitem}
